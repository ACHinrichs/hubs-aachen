\documentclass{scrartcl}

\usepackage{titlesec}
\usepackage{enumitem}
\usepackage[ngerman]{babel}

% Begin sections with the paragraph sign
\titleformat{\section}{\normalfont\Large\bfseries}{\S\thesection}{1em}{}

% Enumerations are formatted like (1), (2), ...
\setenumerate[1]{label={(\arabic*)}}

\title{Satzung eines nicht-gemeinnützigen, nicht eingetragenen Vereins}
\author{Hahn- oder Ballsportverein Aachen}

\begin{document}
    \maketitle
    \section{Name, Sitz und Geschäftsjahr}
        \begin{enumerate}
            \item Der Verein führt den Namen \dq Hahn- oder Ballsportverein Aachen\dq.
            \item Der Verein hat seinen Sitz in Aachen, Deutschland.
            \item Geschäftsjahr ist das Kalenderjahr.
        \end{enumerate}
    \section{Zweck des Vereins und Gemeinnützigkeit}
        \begin{enumerate}
            \item \textbf{Steuerbegünstigte Zwecke} \newline
                Der Verein verfolgt keine gemeinnützigen Zwecke.
            \item \textbf{Sonstige Zwecke} \newline
                Der Verein verfolgt die Ausführung von Hahn- und/oder Ballsport als Gruppenbeschäftigung auf internationaler Ebene.
            \item \textbf{Maßnahmen} \newline
                Der Satzungszweck wird insbesondere durch den Vorsitzenden verwirklicht.
            \item \textbf{Gemeinnützigkeit} \newline
                Der Verein ist selbstlos tätig; er verfolgt nicht in erster Linie eigenwirtschaftliche Zwecke. Mittel
                des Vereins dürfen nur für die satzungsmäßigen Zwecke verwendet werden. Die
                Mitglieder*innen erhalten keine Zuwendungen aus Mitteln des Vereins. Es darf keine Person
                durch Ausgaben, die dem Zweck des Vereins fremd sind, oder durch unverhältnismäßig hohe
                Vergütungen begünstigt werden.
        \end{enumerate}
    \section{Erwerb der Mitgliedschaft und Mitgliedsbeiträge}
        \begin{enumerate}
            \item \textbf{Art der Mitglieder} \newline
                Mitglied des Vereins kann jede natürliche Person werden.
            \item \textbf{Erwerb der Mitgliedschaft} \newline
                Der Antrag auf Mitgliedschaft ist schriftlich an den Vorstand zu stellen. 
                Über die Aufnahme entscheiden alle Gründungsmitglieder gemeinschaftlich durch Wahl in einer außerordentlichen Mitgliederversammlung, siehe \S 7.
                Die Ablehnung bedarf keiner Begründung. 
                Gegen die Ablehnung steht dem Bewerber kein Rechtsmittel zu.
                Bei Aufnahme ist zeitnah™ ein einmaliger Mitgliedsausweis zu fertigen. Dieser Mitgliedsausweis ist nur laminiert gültig. 
            \item \textbf{Beiträge} \newline
                Mitglieder entrichten keinen Jahresbeitrag. 
                Eine Aufwandentschädigung ist in Form eines Kasten beliebigen Bieres oder einer Flasche feinster Spirituosen vorzubringen. 
        \end{enumerate}
    \section{Beendigung der Mitgliedschaft}
        \begin{enumerate}
            \item \textbf{Grund}
                \begin{itemize}
                    \item bei natürlichen Personen durch deren Tod oder Verlust der Geschäftsfähigkeit;
                    \item durch Austritt;
                    \item durch Ausschluss.
                \end{itemize}
            \item \textbf{Austritt} \newline
                Der Austritt eines Mitgliedes erfolgt durch schriftliche Erklärung gegenüber dem
                Vorstand. Der Austritt ist jederzeit möglich.
            \item \textbf{Ausschluss} \newline
                Ein Mitglied kann durch Beschluss des Vorstands mit sofortiger Wirkung aus
                wichtigem Grund aus dem Verein ausgeschlossen werden, wenn ein wichtiger Grund die
                Fortführung der Mitgliedschaft für den Verein oder seine Mitglieder*innen unzumutbar
                erscheinen lässt.
                Ein solcher wichtiger Grund liegt insbesondere dann vor, wenn das Mitglied trotz Mahnung
                länger als sechs Monate mit seiner Beitragszahlung im Rückstand ist oder den
                Vereinsinteressen grob zuwidergehandelt hat.
                Dem Mitglied ist vor seinem Ausschluss Gelegenheit zur Anhörung zu geben.
                Das Mitglied kann gegen den Ausschluss innerhalb einer Frist von einem Monat nach dem
                Zugang der Ausschlusserklärung die nächste ordentliche Mitgliederversammlung anrufen,
                die dann abschließend entscheidet.
            \item \textbf{Pflichten der Mitglieder} \newline
                Mit dem Antrag auf Mitgliedschaft erkennen die Mitglieder den Inhalt der Satzung und der
                sonstigen Vereinsordnungen an. Die Mitglieder sind verpflichtet, die Ziele und Interessen des
                Vereins zu unterstützen sowie die Beschlüsse und Anordnungen der Vereinsorgane zu
                befolgen.
                Die Mitglieder sind verpflichtet, dem Vorstand eine ladungsfähige postalische Anschrift sowie
                eine E-Mail-Adresse mitzuteilen und den Vorstand über jede Änderung ihres Namens und/oder
                ihrer Adressdaten unverzüglich zu informieren.
        \end{enumerate}
    \section{Die Organe des Vereins}
        Die Organe des Vereins sind
        \begin{itemize}
            \item die Mitgliederversammlung;
            \item der Vorstand.
        \end{itemize}
        \begin{enumerate}
            \item \textbf{Anzahl der Vorstandsmitglieder} \newline
                Der Vorstand besteht aus dem 1. Vorsitzenden, dem Abteilungsleiter Ball und dem Abteilungsleiter Hahn.
            \item \textbf{Vertretungsberechtigung} \newline
                Vertretungsberechtigt sind alle Gründungsmitglieder. 
                Durch Beschluss der Mitgliederversammlung können Vorstandsmitglieder von den Beschränkungen des § 181 BGB befreit werden.
            \item \textbf{Aufgaben} \newline
                Der Vorstand führt die Geschäfte und vertritt den Verein in sämtlichen Angelegenheiten
                gerichtlich und außergerichtlich. Darüber hinaus hat er insbesondere folgende Aufgaben:
                \begin{itemize}
                    \item Vorbereitung und Einberufung der Mitgliederversammlung;
                    \item Aufstellen der Tagesordnung;
                    \item Ausführung der Beschlüsse der Mitgliederversammlung;
                    \item Führen der Bücher.
                \end{itemize}
                Der Vorstand hat bei der täglichen \dq Hahn oder Ball? \dq-Wahl ein 1,5-faches Stimmrecht und kann so bei Gleichstand die Wahl zu seinen Gunsten beeinflussen.
            \item \textbf{Wahl} \newline
                Die Vorstandsmitglieder werden von der Mitgliederversammlung für die Dauer von 1 Jahr
                gewählt. Wiederwahl ist möglich. Der Vorstand wird von der
                Mitgliederversammlung im gesonderten Wahlgang bestimmt. 
                Das jeweils amtierenden Vorstandsmitglieder bleibt nach Ablauf ihrer Amtszeit so lange im Amt, bis
                sein Nachfolger gewählt ist. 
                Scheidet der Vorstand vor Ablauf der Amtszeit aus, muss eine sofortige Neuwahl im einer außergewöhnlichen Mitgliederversammlung gewählt werden.
            \item \textbf{Vergütung} \newline
                Der Vorstand ist grundsätzlich ehrenamtlich tätig.
            \item \textbf{Haftungsbeschränkung} \newline    
                Der Vorstand haftet dem Verein gegenüber nur für vorsätzliches oder grob
                fahrlässiges Verhalten. Wird der Vorstand aufgrund seiner Vorstandstätigkeit von
                Dritten in Anspruch genommen, stellt der Verein den Vorstand von
                diesen Ansprüchen frei, sofern dieser nicht vorsätzlich oder grob fahrlässig
                handelte.
        \end{enumerate}
    \section{Ordentliche Mitgliederversammlung}
        \begin{enumerate}
            \item \textbf{Häufigkeit} \newline
                Die Mitgliederversammlung findet mindestens einmal jährlich statt.
            \item \textbf{Präsenzversammlung und virtuelle Mitgliederversammlung} \newline
                Die Mitgliederversammlung kann als Präsenzversammlung oder als virtuelle
                Mitgliederversammlung abgehalten werden. Zur Präsenzversammlung treffen sich alle
                Teilnehmer*innen der Mitgliederversammlung an einem gemeinsamen Ort. Die virtuelle
                Mitgliederversammlung erfolgt durch Einwahl aller Teilnehmer*innen in eine Video- oder
                Telefonkonferenz. Eine Kombination von Präsenzversammlung und virtueller
                Mitgliederversammlung ist möglich, indem den Mitgliedern die Möglichkeit eröffnet wird, an der
                Präsenzversammlung mittels Video- oder Telefonkonferenz teilzunehmen. Der Vorstand
                entscheidet über die Form der Mitgliederversammlung und teilt diese in der Einladung zur
                Mitgliederversammlung mit. Lädt der Vorstand zu einer virtuellen Mitgliederversammlung ein,
                so teilt er den Mitgliedern spätestens eine Stunde vor Beginn der Mitgliederversammlung per
                E-Mail die Einwahldaten für die Video- oder Telefonkonferenz mit.
                Bei Präsenzversammlungen ist der jährlich fällige Mitgliederbeitrag zu entrichten.
            \item \textbf{Einberufung und Tagesordnung} \newline
                Die Einberufung der Mitgliederversammlung erfolgt schriftlich oder per E-Mail durch den
                Vorstand unter Angabe der Tagesordnung mit einer Einladungsfrist von zwei Wochen. Die
                Frist beginnt am Tage der Versendung der Einladung.
                Insbesondere ist eine Einladung als iCal zu versenden.
                Das Einladungsschreiben gilt dem Mitglied als zugegangen, wenn es an die letzte vom Mitglied dem Verein bekannt gegebene
                Adresse gerichtet ist.
                Anträge zur Ergänzung der Tagesordnung können von jedem Mitglied eingebracht werden.
                Sie müssen eine Woche vor der Versammlung dem Vorstand schriftlich mit Begründung
                vorliegen. Der Versammlungsleiter hat die Ergänzung zu Beginn der Versammlung bekannt
                zu geben.
            \item \textbf{Beschlussfähigkeit} \newline
                Die Mitgliederversammlung bei Anwesenheit von 2/3 der Mitglieder beschlussfähig.
            \item \textbf{Beschlussfassung} \newline
                Die Beschlüsse werden mit einfacher Mehrheit der abgegebenen Stimmen gefasst; bei
                Stimmengleichheit entscheidet der 1. Vorsitzende. Stimmenthaltungen gelten als nicht
                abgegebene Stimmen.
                Zur Änderung der Satzung und zur Auflösung des Vereins ist eine Mehrheit von 2/3 der
                abgegebenen Stimmen erforderlich.
                Jedes Mitglied hat eine Stimme. Das Stimmrecht kann nur persönlich oder für ein Mitglied
                unter Vorlage einer schriftlichen Vollmacht ausgeübt werden.
                Über die Beschlüsse der Mitgliederversammlung ist ein Protokoll aufzunehmen, das vom
                jeweiligen Versammlungsleiter und dem Protokollführer zu unterzeichnen ist.
            \item \textbf{Wahlen} \newline
                Für Wahlen gilt Folgendes: Hat im ersten Wahlgang kein*e Kandidat*in die Mehrheit der
                abgegebenen gültigen Stimmen erreicht, findet eine Stichwahl zwischen den Kandidaten statt,
                welche die beiden höchsten Stimmenzahlen erreicht haben.
            \item \textbf{Aufgabenbereiche} \newline
                Die Mitgliederversammlung ist zuständig für
                \begin{itemize}
                    \item die Wahl und Abberufung des Vorstands;
                    \item die Entgegennahme des Jahresberichts und die Entlastung des Vorstands;
                    \item die Festsetzung der Höhe und der Fälligkeit des Jahresbeitrages (eventuell Auslagerung in Gebührenordnung);
                    \item die Beschlussfassung über Satzungsänderungen und die Auflösung des Vereins.
                \end{itemize}
            \item \textbf{Versammlungsleitung} \newline  
                Die Mitgliederversammlung wird vom Vorstand geleitet. Ist dieser nicht anwesend, so bestimmt die Versammlung den Leiter mit einfacher
                Mehrheit der abgegebenen Stimmen. Der Versammlungsleiter bestimmt einen Protokollführer.    
        \end{enumerate}
    \section{Außerordentliche Mitgliederversammlung}
        Eine außerordentliche Mitgliederversammlung findet statt, wenn das Interesse des
        Vereins es erfordert oder wenn 1/5 der Mitglieder es schriftlich unter Angabe der Gründe beim
        Vorstand beantragt.
        Es gilt \S 6 Absatz (2)-(8).
    \section{Auflösung des Vereins}
        Bei Auflösung oder Aufhebung des Vereins wird das Vermögen des Vereins unter allen Mitgliedern gleich aufgeteilt.

\end{document}
